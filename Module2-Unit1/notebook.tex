
% Default to the notebook output style

    


% Inherit from the specified cell style.




    
\documentclass[11pt]{article}

    
    
    \usepackage[T1]{fontenc}
    % Nicer default font (+ math font) than Computer Modern for most use cases
    \usepackage{mathpazo}

    % Basic figure setup, for now with no caption control since it's done
    % automatically by Pandoc (which extracts ![](path) syntax from Markdown).
    \usepackage{graphicx}
    % We will generate all images so they have a width \maxwidth. This means
    % that they will get their normal width if they fit onto the page, but
    % are scaled down if they would overflow the margins.
    \makeatletter
    \def\maxwidth{\ifdim\Gin@nat@width>\linewidth\linewidth
    \else\Gin@nat@width\fi}
    \makeatother
    \let\Oldincludegraphics\includegraphics
    % Set max figure width to be 80% of text width, for now hardcoded.
    \renewcommand{\includegraphics}[1]{\Oldincludegraphics[width=.8\maxwidth]{#1}}
    % Ensure that by default, figures have no caption (until we provide a
    % proper Figure object with a Caption API and a way to capture that
    % in the conversion process - todo).
    \usepackage{caption}
    \DeclareCaptionLabelFormat{nolabel}{}
    \captionsetup{labelformat=nolabel}

    \usepackage{adjustbox} % Used to constrain images to a maximum size 
    \usepackage{xcolor} % Allow colors to be defined
    \usepackage{enumerate} % Needed for markdown enumerations to work
    \usepackage{geometry} % Used to adjust the document margins
    \usepackage{amsmath} % Equations
    \usepackage{amssymb} % Equations
    \usepackage{textcomp} % defines textquotesingle
    % Hack from http://tex.stackexchange.com/a/47451/13684:
    \AtBeginDocument{%
        \def\PYZsq{\textquotesingle}% Upright quotes in Pygmentized code
    }
    \usepackage{upquote} % Upright quotes for verbatim code
    \usepackage{eurosym} % defines \euro
    \usepackage[mathletters]{ucs} % Extended unicode (utf-8) support
    \usepackage[utf8x]{inputenc} % Allow utf-8 characters in the tex document
    \usepackage{fancyvrb} % verbatim replacement that allows latex
    \usepackage{grffile} % extends the file name processing of package graphics 
                         % to support a larger range 
    % The hyperref package gives us a pdf with properly built
    % internal navigation ('pdf bookmarks' for the table of contents,
    % internal cross-reference links, web links for URLs, etc.)
    \usepackage{hyperref}
    \usepackage{longtable} % longtable support required by pandoc >1.10
    \usepackage{booktabs}  % table support for pandoc > 1.12.2
    \usepackage[inline]{enumitem} % IRkernel/repr support (it uses the enumerate* environment)
    \usepackage[normalem]{ulem} % ulem is needed to support strikethroughs (\sout)
                                % normalem makes italics be italics, not underlines
    

    
    
    % Colors for the hyperref package
    \definecolor{urlcolor}{rgb}{0,.145,.698}
    \definecolor{linkcolor}{rgb}{.71,0.21,0.01}
    \definecolor{citecolor}{rgb}{.12,.54,.11}

    % ANSI colors
    \definecolor{ansi-black}{HTML}{3E424D}
    \definecolor{ansi-black-intense}{HTML}{282C36}
    \definecolor{ansi-red}{HTML}{E75C58}
    \definecolor{ansi-red-intense}{HTML}{B22B31}
    \definecolor{ansi-green}{HTML}{00A250}
    \definecolor{ansi-green-intense}{HTML}{007427}
    \definecolor{ansi-yellow}{HTML}{DDB62B}
    \definecolor{ansi-yellow-intense}{HTML}{B27D12}
    \definecolor{ansi-blue}{HTML}{208FFB}
    \definecolor{ansi-blue-intense}{HTML}{0065CA}
    \definecolor{ansi-magenta}{HTML}{D160C4}
    \definecolor{ansi-magenta-intense}{HTML}{A03196}
    \definecolor{ansi-cyan}{HTML}{60C6C8}
    \definecolor{ansi-cyan-intense}{HTML}{258F8F}
    \definecolor{ansi-white}{HTML}{C5C1B4}
    \definecolor{ansi-white-intense}{HTML}{A1A6B2}

    % commands and environments needed by pandoc snippets
    % extracted from the output of `pandoc -s`
    \providecommand{\tightlist}{%
      \setlength{\itemsep}{0pt}\setlength{\parskip}{0pt}}
    \DefineVerbatimEnvironment{Highlighting}{Verbatim}{commandchars=\\\{\}}
    % Add ',fontsize=\small' for more characters per line
    \newenvironment{Shaded}{}{}
    \newcommand{\KeywordTok}[1]{\textcolor[rgb]{0.00,0.44,0.13}{\textbf{{#1}}}}
    \newcommand{\DataTypeTok}[1]{\textcolor[rgb]{0.56,0.13,0.00}{{#1}}}
    \newcommand{\DecValTok}[1]{\textcolor[rgb]{0.25,0.63,0.44}{{#1}}}
    \newcommand{\BaseNTok}[1]{\textcolor[rgb]{0.25,0.63,0.44}{{#1}}}
    \newcommand{\FloatTok}[1]{\textcolor[rgb]{0.25,0.63,0.44}{{#1}}}
    \newcommand{\CharTok}[1]{\textcolor[rgb]{0.25,0.44,0.63}{{#1}}}
    \newcommand{\StringTok}[1]{\textcolor[rgb]{0.25,0.44,0.63}{{#1}}}
    \newcommand{\CommentTok}[1]{\textcolor[rgb]{0.38,0.63,0.69}{\textit{{#1}}}}
    \newcommand{\OtherTok}[1]{\textcolor[rgb]{0.00,0.44,0.13}{{#1}}}
    \newcommand{\AlertTok}[1]{\textcolor[rgb]{1.00,0.00,0.00}{\textbf{{#1}}}}
    \newcommand{\FunctionTok}[1]{\textcolor[rgb]{0.02,0.16,0.49}{{#1}}}
    \newcommand{\RegionMarkerTok}[1]{{#1}}
    \newcommand{\ErrorTok}[1]{\textcolor[rgb]{1.00,0.00,0.00}{\textbf{{#1}}}}
    \newcommand{\NormalTok}[1]{{#1}}
    
    % Additional commands for more recent versions of Pandoc
    \newcommand{\ConstantTok}[1]{\textcolor[rgb]{0.53,0.00,0.00}{{#1}}}
    \newcommand{\SpecialCharTok}[1]{\textcolor[rgb]{0.25,0.44,0.63}{{#1}}}
    \newcommand{\VerbatimStringTok}[1]{\textcolor[rgb]{0.25,0.44,0.63}{{#1}}}
    \newcommand{\SpecialStringTok}[1]{\textcolor[rgb]{0.73,0.40,0.53}{{#1}}}
    \newcommand{\ImportTok}[1]{{#1}}
    \newcommand{\DocumentationTok}[1]{\textcolor[rgb]{0.73,0.13,0.13}{\textit{{#1}}}}
    \newcommand{\AnnotationTok}[1]{\textcolor[rgb]{0.38,0.63,0.69}{\textbf{\textit{{#1}}}}}
    \newcommand{\CommentVarTok}[1]{\textcolor[rgb]{0.38,0.63,0.69}{\textbf{\textit{{#1}}}}}
    \newcommand{\VariableTok}[1]{\textcolor[rgb]{0.10,0.09,0.49}{{#1}}}
    \newcommand{\ControlFlowTok}[1]{\textcolor[rgb]{0.00,0.44,0.13}{\textbf{{#1}}}}
    \newcommand{\OperatorTok}[1]{\textcolor[rgb]{0.40,0.40,0.40}{{#1}}}
    \newcommand{\BuiltInTok}[1]{{#1}}
    \newcommand{\ExtensionTok}[1]{{#1}}
    \newcommand{\PreprocessorTok}[1]{\textcolor[rgb]{0.74,0.48,0.00}{{#1}}}
    \newcommand{\AttributeTok}[1]{\textcolor[rgb]{0.49,0.56,0.16}{{#1}}}
    \newcommand{\InformationTok}[1]{\textcolor[rgb]{0.38,0.63,0.69}{\textbf{\textit{{#1}}}}}
    \newcommand{\WarningTok}[1]{\textcolor[rgb]{0.38,0.63,0.69}{\textbf{\textit{{#1}}}}}
    
    
    % Define a nice break command that doesn't care if a line doesn't already
    % exist.
    \def\br{\hspace*{\fill} \\* }
    % Math Jax compatability definitions
    \def\gt{>}
    \def\lt{<}
    % Document parameters
    \title{Introduction}
    
    
    

    % Pygments definitions
    
\makeatletter
\def\PY@reset{\let\PY@it=\relax \let\PY@bf=\relax%
    \let\PY@ul=\relax \let\PY@tc=\relax%
    \let\PY@bc=\relax \let\PY@ff=\relax}
\def\PY@tok#1{\csname PY@tok@#1\endcsname}
\def\PY@toks#1+{\ifx\relax#1\empty\else%
    \PY@tok{#1}\expandafter\PY@toks\fi}
\def\PY@do#1{\PY@bc{\PY@tc{\PY@ul{%
    \PY@it{\PY@bf{\PY@ff{#1}}}}}}}
\def\PY#1#2{\PY@reset\PY@toks#1+\relax+\PY@do{#2}}

\expandafter\def\csname PY@tok@w\endcsname{\def\PY@tc##1{\textcolor[rgb]{0.73,0.73,0.73}{##1}}}
\expandafter\def\csname PY@tok@c\endcsname{\let\PY@it=\textit\def\PY@tc##1{\textcolor[rgb]{0.25,0.50,0.50}{##1}}}
\expandafter\def\csname PY@tok@cp\endcsname{\def\PY@tc##1{\textcolor[rgb]{0.74,0.48,0.00}{##1}}}
\expandafter\def\csname PY@tok@k\endcsname{\let\PY@bf=\textbf\def\PY@tc##1{\textcolor[rgb]{0.00,0.50,0.00}{##1}}}
\expandafter\def\csname PY@tok@kp\endcsname{\def\PY@tc##1{\textcolor[rgb]{0.00,0.50,0.00}{##1}}}
\expandafter\def\csname PY@tok@kt\endcsname{\def\PY@tc##1{\textcolor[rgb]{0.69,0.00,0.25}{##1}}}
\expandafter\def\csname PY@tok@o\endcsname{\def\PY@tc##1{\textcolor[rgb]{0.40,0.40,0.40}{##1}}}
\expandafter\def\csname PY@tok@ow\endcsname{\let\PY@bf=\textbf\def\PY@tc##1{\textcolor[rgb]{0.67,0.13,1.00}{##1}}}
\expandafter\def\csname PY@tok@nb\endcsname{\def\PY@tc##1{\textcolor[rgb]{0.00,0.50,0.00}{##1}}}
\expandafter\def\csname PY@tok@nf\endcsname{\def\PY@tc##1{\textcolor[rgb]{0.00,0.00,1.00}{##1}}}
\expandafter\def\csname PY@tok@nc\endcsname{\let\PY@bf=\textbf\def\PY@tc##1{\textcolor[rgb]{0.00,0.00,1.00}{##1}}}
\expandafter\def\csname PY@tok@nn\endcsname{\let\PY@bf=\textbf\def\PY@tc##1{\textcolor[rgb]{0.00,0.00,1.00}{##1}}}
\expandafter\def\csname PY@tok@ne\endcsname{\let\PY@bf=\textbf\def\PY@tc##1{\textcolor[rgb]{0.82,0.25,0.23}{##1}}}
\expandafter\def\csname PY@tok@nv\endcsname{\def\PY@tc##1{\textcolor[rgb]{0.10,0.09,0.49}{##1}}}
\expandafter\def\csname PY@tok@no\endcsname{\def\PY@tc##1{\textcolor[rgb]{0.53,0.00,0.00}{##1}}}
\expandafter\def\csname PY@tok@nl\endcsname{\def\PY@tc##1{\textcolor[rgb]{0.63,0.63,0.00}{##1}}}
\expandafter\def\csname PY@tok@ni\endcsname{\let\PY@bf=\textbf\def\PY@tc##1{\textcolor[rgb]{0.60,0.60,0.60}{##1}}}
\expandafter\def\csname PY@tok@na\endcsname{\def\PY@tc##1{\textcolor[rgb]{0.49,0.56,0.16}{##1}}}
\expandafter\def\csname PY@tok@nt\endcsname{\let\PY@bf=\textbf\def\PY@tc##1{\textcolor[rgb]{0.00,0.50,0.00}{##1}}}
\expandafter\def\csname PY@tok@nd\endcsname{\def\PY@tc##1{\textcolor[rgb]{0.67,0.13,1.00}{##1}}}
\expandafter\def\csname PY@tok@s\endcsname{\def\PY@tc##1{\textcolor[rgb]{0.73,0.13,0.13}{##1}}}
\expandafter\def\csname PY@tok@sd\endcsname{\let\PY@it=\textit\def\PY@tc##1{\textcolor[rgb]{0.73,0.13,0.13}{##1}}}
\expandafter\def\csname PY@tok@si\endcsname{\let\PY@bf=\textbf\def\PY@tc##1{\textcolor[rgb]{0.73,0.40,0.53}{##1}}}
\expandafter\def\csname PY@tok@se\endcsname{\let\PY@bf=\textbf\def\PY@tc##1{\textcolor[rgb]{0.73,0.40,0.13}{##1}}}
\expandafter\def\csname PY@tok@sr\endcsname{\def\PY@tc##1{\textcolor[rgb]{0.73,0.40,0.53}{##1}}}
\expandafter\def\csname PY@tok@ss\endcsname{\def\PY@tc##1{\textcolor[rgb]{0.10,0.09,0.49}{##1}}}
\expandafter\def\csname PY@tok@sx\endcsname{\def\PY@tc##1{\textcolor[rgb]{0.00,0.50,0.00}{##1}}}
\expandafter\def\csname PY@tok@m\endcsname{\def\PY@tc##1{\textcolor[rgb]{0.40,0.40,0.40}{##1}}}
\expandafter\def\csname PY@tok@gh\endcsname{\let\PY@bf=\textbf\def\PY@tc##1{\textcolor[rgb]{0.00,0.00,0.50}{##1}}}
\expandafter\def\csname PY@tok@gu\endcsname{\let\PY@bf=\textbf\def\PY@tc##1{\textcolor[rgb]{0.50,0.00,0.50}{##1}}}
\expandafter\def\csname PY@tok@gd\endcsname{\def\PY@tc##1{\textcolor[rgb]{0.63,0.00,0.00}{##1}}}
\expandafter\def\csname PY@tok@gi\endcsname{\def\PY@tc##1{\textcolor[rgb]{0.00,0.63,0.00}{##1}}}
\expandafter\def\csname PY@tok@gr\endcsname{\def\PY@tc##1{\textcolor[rgb]{1.00,0.00,0.00}{##1}}}
\expandafter\def\csname PY@tok@ge\endcsname{\let\PY@it=\textit}
\expandafter\def\csname PY@tok@gs\endcsname{\let\PY@bf=\textbf}
\expandafter\def\csname PY@tok@gp\endcsname{\let\PY@bf=\textbf\def\PY@tc##1{\textcolor[rgb]{0.00,0.00,0.50}{##1}}}
\expandafter\def\csname PY@tok@go\endcsname{\def\PY@tc##1{\textcolor[rgb]{0.53,0.53,0.53}{##1}}}
\expandafter\def\csname PY@tok@gt\endcsname{\def\PY@tc##1{\textcolor[rgb]{0.00,0.27,0.87}{##1}}}
\expandafter\def\csname PY@tok@err\endcsname{\def\PY@bc##1{\setlength{\fboxsep}{0pt}\fcolorbox[rgb]{1.00,0.00,0.00}{1,1,1}{\strut ##1}}}
\expandafter\def\csname PY@tok@kc\endcsname{\let\PY@bf=\textbf\def\PY@tc##1{\textcolor[rgb]{0.00,0.50,0.00}{##1}}}
\expandafter\def\csname PY@tok@kd\endcsname{\let\PY@bf=\textbf\def\PY@tc##1{\textcolor[rgb]{0.00,0.50,0.00}{##1}}}
\expandafter\def\csname PY@tok@kn\endcsname{\let\PY@bf=\textbf\def\PY@tc##1{\textcolor[rgb]{0.00,0.50,0.00}{##1}}}
\expandafter\def\csname PY@tok@kr\endcsname{\let\PY@bf=\textbf\def\PY@tc##1{\textcolor[rgb]{0.00,0.50,0.00}{##1}}}
\expandafter\def\csname PY@tok@bp\endcsname{\def\PY@tc##1{\textcolor[rgb]{0.00,0.50,0.00}{##1}}}
\expandafter\def\csname PY@tok@fm\endcsname{\def\PY@tc##1{\textcolor[rgb]{0.00,0.00,1.00}{##1}}}
\expandafter\def\csname PY@tok@vc\endcsname{\def\PY@tc##1{\textcolor[rgb]{0.10,0.09,0.49}{##1}}}
\expandafter\def\csname PY@tok@vg\endcsname{\def\PY@tc##1{\textcolor[rgb]{0.10,0.09,0.49}{##1}}}
\expandafter\def\csname PY@tok@vi\endcsname{\def\PY@tc##1{\textcolor[rgb]{0.10,0.09,0.49}{##1}}}
\expandafter\def\csname PY@tok@vm\endcsname{\def\PY@tc##1{\textcolor[rgb]{0.10,0.09,0.49}{##1}}}
\expandafter\def\csname PY@tok@sa\endcsname{\def\PY@tc##1{\textcolor[rgb]{0.73,0.13,0.13}{##1}}}
\expandafter\def\csname PY@tok@sb\endcsname{\def\PY@tc##1{\textcolor[rgb]{0.73,0.13,0.13}{##1}}}
\expandafter\def\csname PY@tok@sc\endcsname{\def\PY@tc##1{\textcolor[rgb]{0.73,0.13,0.13}{##1}}}
\expandafter\def\csname PY@tok@dl\endcsname{\def\PY@tc##1{\textcolor[rgb]{0.73,0.13,0.13}{##1}}}
\expandafter\def\csname PY@tok@s2\endcsname{\def\PY@tc##1{\textcolor[rgb]{0.73,0.13,0.13}{##1}}}
\expandafter\def\csname PY@tok@sh\endcsname{\def\PY@tc##1{\textcolor[rgb]{0.73,0.13,0.13}{##1}}}
\expandafter\def\csname PY@tok@s1\endcsname{\def\PY@tc##1{\textcolor[rgb]{0.73,0.13,0.13}{##1}}}
\expandafter\def\csname PY@tok@mb\endcsname{\def\PY@tc##1{\textcolor[rgb]{0.40,0.40,0.40}{##1}}}
\expandafter\def\csname PY@tok@mf\endcsname{\def\PY@tc##1{\textcolor[rgb]{0.40,0.40,0.40}{##1}}}
\expandafter\def\csname PY@tok@mh\endcsname{\def\PY@tc##1{\textcolor[rgb]{0.40,0.40,0.40}{##1}}}
\expandafter\def\csname PY@tok@mi\endcsname{\def\PY@tc##1{\textcolor[rgb]{0.40,0.40,0.40}{##1}}}
\expandafter\def\csname PY@tok@il\endcsname{\def\PY@tc##1{\textcolor[rgb]{0.40,0.40,0.40}{##1}}}
\expandafter\def\csname PY@tok@mo\endcsname{\def\PY@tc##1{\textcolor[rgb]{0.40,0.40,0.40}{##1}}}
\expandafter\def\csname PY@tok@ch\endcsname{\let\PY@it=\textit\def\PY@tc##1{\textcolor[rgb]{0.25,0.50,0.50}{##1}}}
\expandafter\def\csname PY@tok@cm\endcsname{\let\PY@it=\textit\def\PY@tc##1{\textcolor[rgb]{0.25,0.50,0.50}{##1}}}
\expandafter\def\csname PY@tok@cpf\endcsname{\let\PY@it=\textit\def\PY@tc##1{\textcolor[rgb]{0.25,0.50,0.50}{##1}}}
\expandafter\def\csname PY@tok@c1\endcsname{\let\PY@it=\textit\def\PY@tc##1{\textcolor[rgb]{0.25,0.50,0.50}{##1}}}
\expandafter\def\csname PY@tok@cs\endcsname{\let\PY@it=\textit\def\PY@tc##1{\textcolor[rgb]{0.25,0.50,0.50}{##1}}}

\def\PYZbs{\char`\\}
\def\PYZus{\char`\_}
\def\PYZob{\char`\{}
\def\PYZcb{\char`\}}
\def\PYZca{\char`\^}
\def\PYZam{\char`\&}
\def\PYZlt{\char`\<}
\def\PYZgt{\char`\>}
\def\PYZsh{\char`\#}
\def\PYZpc{\char`\%}
\def\PYZdl{\char`\$}
\def\PYZhy{\char`\-}
\def\PYZsq{\char`\'}
\def\PYZdq{\char`\"}
\def\PYZti{\char`\~}
% for compatibility with earlier versions
\def\PYZat{@}
\def\PYZlb{[}
\def\PYZrb{]}
\makeatother


    % Exact colors from NB
    \definecolor{incolor}{rgb}{0.0, 0.0, 0.5}
    \definecolor{outcolor}{rgb}{0.545, 0.0, 0.0}



    
    % Prevent overflowing lines due to hard-to-break entities
    \sloppy 
    % Setup hyperref package
    \hypersetup{
      breaklinks=true,  % so long urls are correctly broken across lines
      colorlinks=true,
      urlcolor=urlcolor,
      linkcolor=linkcolor,
      citecolor=citecolor,
      }
    % Slightly bigger margins than the latex defaults
    
    \geometry{verbose,tmargin=1in,bmargin=1in,lmargin=1in,rmargin=1in}
    
    

    \begin{document}
    
    
    \maketitle
    
    

    
    \hypertarget{assignment-1.1-python}{%
\section{Assignment 1.1 Python}\label{assignment-1.1-python}}

Python is an easy to learn, powerful programming language with efficient
high-level data structures and object-oriented programming. Python's
elegant syntax and dynamic typing, together with its interpreted nature,
make it an ideal language for scripting and rapid application
development in many areas on most platforms.

    \hypertarget{jupyter-notebook}{%
\subsubsection{Jupyter Notebook}\label{jupyter-notebook}}

What you are reading now is an example of a Jupyter Notebook. The basic
concept is that of a ``notebook'' containing text and programming code.
You can easily edit the notebook using your web browser, run the
programs in the ipython server in the background and see the output of
the programs within the notebook. This is a powerful paradigm that is
well suited to machine learning research, particularly when
collaborating with other people. For Reference: -
https://ipython.org/notebook.html - https://jupyter.org/

    \begin{Verbatim}[commandchars=\\\{\}]
{\color{incolor}In [{\color{incolor}1}]:} \PY{c+c1}{\PYZsh{} This rectangular box is a \PYZdq{}Cell\PYZdq{} in Jupyter Notebook (click to edit)}
        \PY{c+c1}{\PYZsh{} This line starting with \PYZsh{} symbol is a comment}
        \PY{c+c1}{\PYZsh{} Press run on the menu or CTRL + ENTER to run this cell}
        \PY{c+c1}{\PYZsh{} or SHIFT + ENTER to run this cell and move to next cell}
        \PY{c+c1}{\PYZsh{} to escape from the current cell press ESC and H to see shortcut keys available}
        
        \PY{n}{var1} \PY{o}{=} \PY{l+m+mi}{5}
        \PY{n}{var2} \PY{o}{=} \PY{l+m+mi}{6}
        \PY{n+nb}{print}\PY{p}{(}\PY{n}{var1} \PY{o}{*} \PY{n}{var2}\PY{p}{)}
\end{Verbatim}


    \begin{Verbatim}[commandchars=\\\{\}]
30

    \end{Verbatim}

    Please go through https://docs.python.org/3/tutorial/introduction.html
to get started with basic python introduction Python also provides some
built-in data types, in particular, dict, list, set and frozenset, and
tuple. The str class is used to hold Unicode strings, and the bytes
class is used to hold binary data. Visit the following page to know
about the built-in data types
https://docs.python.org/3/library/stdtypes.html

    \hypertarget{tuples}{%
\subsubsection{TUPLES}\label{tuples}}

    \begin{Verbatim}[commandchars=\\\{\}]
{\color{incolor}In [{\color{incolor}2}]:} \PY{c+c1}{\PYZsh{} Experiement here with your knowledge of Tuples}
\end{Verbatim}


    \hypertarget{exercise-1}{%
\subsubsection{Exercise 1}\label{exercise-1}}

T is a Nested Tuple, use indexing to extract `g' from the Tuple

    \begin{Verbatim}[commandchars=\\\{\}]
{\color{incolor}In [{\color{incolor}3}]:} \PY{n}{T} \PY{o}{=} \PY{p}{(}\PY{l+s+s1}{\PYZsq{}}\PY{l+s+s1}{a}\PY{l+s+s1}{\PYZsq{}}\PY{p}{,}\PY{l+s+s1}{\PYZsq{}}\PY{l+s+s1}{b}\PY{l+s+s1}{\PYZsq{}}\PY{p}{,}\PY{l+s+s1}{\PYZsq{}}\PY{l+s+s1}{c}\PY{l+s+s1}{\PYZsq{}}\PY{p}{,}\PY{p}{(}\PY{l+s+s1}{\PYZsq{}}\PY{l+s+s1}{d}\PY{l+s+s1}{\PYZsq{}}\PY{p}{,}\PY{l+s+s1}{\PYZsq{}}\PY{l+s+s1}{e}\PY{l+s+s1}{\PYZsq{}}\PY{p}{)}\PY{p}{,}\PY{p}{(}\PY{l+s+s1}{\PYZsq{}}\PY{l+s+s1}{f}\PY{l+s+s1}{\PYZsq{}}\PY{p}{,}\PY{p}{(}\PY{l+s+s1}{\PYZsq{}}\PY{l+s+s1}{f}\PY{l+s+s1}{\PYZsq{}}\PY{p}{,}\PY{l+s+s1}{\PYZsq{}}\PY{l+s+s1}{g}\PY{l+s+s1}{\PYZsq{}}\PY{p}{)}\PY{p}{,}\PY{l+s+s1}{\PYZsq{}}\PY{l+s+s1}{k}\PY{l+s+s1}{\PYZsq{}}\PY{p}{)}\PY{p}{,}\PY{l+s+s1}{\PYZsq{}}\PY{l+s+s1}{p}\PY{l+s+s1}{\PYZsq{}}\PY{p}{)}
        \PY{n}{thisisg} \PY{o}{=} \PY{k+kc}{None}
        \PY{c+c1}{\PYZsh{} YOUR CODE HERE}
        \PY{n}{thisisg} \PY{o}{=} \PY{n}{T}\PY{p}{[}\PY{l+m+mi}{4}\PY{p}{]}\PY{p}{[}\PY{l+m+mi}{1}\PY{p}{]}\PY{p}{[}\PY{l+m+mi}{1}\PY{p}{]}
\end{Verbatim}


    \begin{Verbatim}[commandchars=\\\{\}]
{\color{incolor}In [{\color{incolor}4}]:} \PY{k}{assert}\PY{p}{(}\PY{n}{thisisg}\PY{o}{==}\PY{l+s+s1}{\PYZsq{}}\PY{l+s+s1}{g}\PY{l+s+s1}{\PYZsq{}}\PY{p}{)}
        \PY{c+c1}{\PYZsh{}\PYZsh{}\PYZsh{} BEGIN HIDDEN hTESTS}
        \PY{k}{assert}\PY{p}{(}\PY{n}{thisisg}\PY{o}{==}\PY{l+s+s1}{\PYZsq{}}\PY{l+s+s1}{g}\PY{l+s+s1}{\PYZsq{}}\PY{p}{)}
\end{Verbatim}


    \hypertarget{exercise-2-lists}{%
\subsubsection{Exercise 2 (Lists)}\label{exercise-2-lists}}

Find the sum of the smallest 10 numbers of List L Indexing, Sorted and
sum function could be useful

    \begin{Verbatim}[commandchars=\\\{\}]
{\color{incolor}In [{\color{incolor}5}]:} \PY{n}{L} \PY{o}{=}\PY{p}{[}\PY{l+m+mi}{75}\PY{p}{,} \PY{l+m+mi}{28}\PY{p}{,} \PY{l+m+mi}{47}\PY{p}{,} \PY{l+m+mi}{51}\PY{p}{,} \PY{l+m+mi}{96}\PY{p}{,} \PY{l+m+mi}{76}\PY{p}{,} \PY{l+m+mi}{32}\PY{p}{,} \PY{l+m+mi}{57}\PY{p}{,} \PY{l+m+mi}{10}\PY{p}{,} \PY{l+m+mi}{10}\PY{p}{,} \PY{l+m+mi}{25}\PY{p}{,} \PY{l+m+mi}{62}\PY{p}{,} \PY{l+m+mi}{17}\PY{p}{,}  \PY{l+m+mi}{9}\PY{p}{,} \PY{l+m+mi}{68}\PY{p}{,} \PY{l+m+mi}{65}\PY{p}{,} \PY{o}{\PYZhy{}}\PY{l+m+mi}{2}\PY{p}{,}
               \PY{l+m+mi}{54}\PY{p}{,} \PY{l+m+mi}{34}\PY{p}{,} \PY{l+m+mi}{41}\PY{p}{,} \PY{l+m+mi}{74}\PY{p}{,} \PY{l+m+mi}{83}\PY{p}{,} \PY{l+m+mi}{91}\PY{p}{,} \PY{l+m+mi}{59}\PY{p}{,} \PY{o}{\PYZhy{}}\PY{l+m+mi}{3}\PY{p}{,} \PY{l+m+mi}{88}\PY{p}{,} \PY{l+m+mi}{59}\PY{p}{,} \PY{l+m+mi}{91}\PY{p}{,} \PY{l+m+mi}{10}\PY{p}{,} \PY{o}{\PYZhy{}}\PY{l+m+mi}{9}\PY{p}{,} \PY{l+m+mi}{90}\PY{p}{,} \PY{l+m+mi}{46}\PY{p}{,} \PY{l+m+mi}{58}\PY{p}{,} \PY{l+m+mi}{25}\PY{p}{,}
               \PY{l+m+mi}{62}\PY{p}{,} \PY{l+m+mi}{32}\PY{p}{,} \PY{l+m+mi}{74}\PY{p}{,}  \PY{l+m+mi}{1}\PY{p}{,} \PY{l+m+mi}{61}\PY{p}{,} \PY{l+m+mi}{43}\PY{p}{,} \PY{l+m+mi}{25}\PY{p}{,} \PY{l+m+mi}{62}\PY{p}{,} \PY{o}{\PYZhy{}}\PY{l+m+mi}{5}\PY{p}{,} \PY{l+m+mi}{49}\PY{p}{,} \PY{l+m+mi}{40}\PY{p}{,} \PY{l+m+mi}{44}\PY{p}{,} \PY{l+m+mi}{83}\PY{p}{,} \PY{o}{\PYZhy{}}\PY{l+m+mi}{5}\PY{p}{,} \PY{l+m+mi}{13}\PY{p}{,} \PY{l+m+mi}{35}\PY{p}{]}
        
        \PY{c+c1}{\PYZsh{} YOUR CODE HERE}
        \PY{n}{ans} \PY{o}{=} \PY{n+nb}{sum}\PY{p}{(}\PY{n+nb}{sorted}\PY{p}{(}\PY{n}{L}\PY{p}{)}\PY{p}{[}\PY{p}{:}\PY{l+m+mi}{10}\PY{p}{]}\PY{p}{)}
\end{Verbatim}


    \begin{Verbatim}[commandchars=\\\{\}]
{\color{incolor}In [{\color{incolor}6}]:} \PY{k}{assert}\PY{p}{(}\PY{n}{ans}\PY{o}{==}\PY{l+m+mi}{16}\PY{p}{)}
\end{Verbatim}


    \hypertarget{control-flow}{%
\subsection{Control Flow}\label{control-flow}}

Python supports different Controls flow statements: - While \ldots{} -
if \ldots{} elif \ldots{} else \ldots{} - for \ldots{} in \ldots{} -
with \ldots{}

https://docs.python.org/3/tutorial/controlflow.html

    \begin{Verbatim}[commandchars=\\\{\}]
{\color{incolor}In [{\color{incolor}7}]:} \PY{c+c1}{\PYZsh{} Experiment with Loops here; try printing out number which are divisible by 3}
        \PY{k}{for} \PY{n}{i} \PY{o+ow}{in} \PY{n+nb}{range}\PY{p}{(}\PY{l+m+mi}{1}\PY{p}{,}\PY{l+m+mi}{100}\PY{p}{)}\PY{p}{:}
            \PY{k}{if} \PY{n}{i}\PY{o}{\PYZpc{}}\PY{k}{3}==0:
                \PY{n+nb}{print}\PY{p}{(}\PY{n}{i}\PY{p}{)}
\end{Verbatim}


    \begin{Verbatim}[commandchars=\\\{\}]
3
6
9
12
15
18
21
24
27
30
33
36
39
42
45
48
51
54
57
60
63
66
69
72
75
78
81
84
87
90
93
96
99

    \end{Verbatim}

    \hypertarget{exercise-3}{%
\subsubsection{Exercise 3}\label{exercise-3}}

Write a program which will list all such numbers which are divisible by
7 but are not a multiple of 5, between 2000 and 3200 (both
included).logical Operator, for, if, and range could be helpful.

    \begin{Verbatim}[commandchars=\\\{\}]
{\color{incolor}In [{\color{incolor}8}]:} \PY{c+c1}{\PYZsh{} YOUR CODE HERE}
        \PY{k}{for} \PY{n}{i} \PY{o+ow}{in} \PY{n+nb}{range}\PY{p}{(}\PY{l+m+mi}{2000}\PY{p}{,}\PY{l+m+mi}{3201}\PY{p}{)}\PY{p}{:}
            \PY{k}{if} \PY{p}{(}\PY{n}{i}\PY{o}{\PYZpc{}}\PY{k}{7}==0 and i\PYZpc{}5!=0):
                \PY{n+nb}{print}\PY{p}{(}\PY{n}{i}\PY{p}{)}
\end{Verbatim}


    \begin{Verbatim}[commandchars=\\\{\}]
2002
2009
2016
2023
2037
2044
2051
2058
2072
2079
2086
2093
2107
2114
2121
2128
2142
2149
2156
2163
2177
2184
2191
2198
2212
2219
2226
2233
2247
2254
2261
2268
2282
2289
2296
2303
2317
2324
2331
2338
2352
2359
2366
2373
2387
2394
2401
2408
2422
2429
2436
2443
2457
2464
2471
2478
2492
2499
2506
2513
2527
2534
2541
2548
2562
2569
2576
2583
2597
2604
2611
2618
2632
2639
2646
2653
2667
2674
2681
2688
2702
2709
2716
2723
2737
2744
2751
2758
2772
2779
2786
2793
2807
2814
2821
2828
2842
2849
2856
2863
2877
2884
2891
2898
2912
2919
2926
2933
2947
2954
2961
2968
2982
2989
2996
3003
3017
3024
3031
3038
3052
3059
3066
3073
3087
3094
3101
3108
3122
3129
3136
3143
3157
3164
3171
3178
3192
3199

    \end{Verbatim}

    \hypertarget{exercise-4-function}{%
\subsubsection{Exercise 4 (Function)}\label{exercise-4-function}}

write a function fibbo to return a list of ``n'' (greater than 2)
fibonnaci number

    \begin{Verbatim}[commandchars=\\\{\}]
{\color{incolor}In [{\color{incolor}9}]:} \PY{k}{def} \PY{n+nf}{fibbo}\PY{p}{(}\PY{n}{n}\PY{p}{)}\PY{p}{:}
        \PY{c+c1}{\PYZsh{} YOUR CODE HERE}
            \PY{n}{fiblist} \PY{o}{=} \PY{p}{[}\PY{l+m+mi}{1}\PY{p}{,} \PY{l+m+mi}{1}\PY{p}{]}
            \PY{n}{a} \PY{o}{=} \PY{l+m+mi}{1}
            \PY{n}{b} \PY{o}{=} \PY{l+m+mi}{1}
            \PY{k}{for} \PY{n}{i} \PY{o+ow}{in} \PY{n+nb}{range}\PY{p}{(}\PY{n}{n}\PY{o}{\PYZhy{}}\PY{l+m+mi}{2}\PY{p}{)}\PY{p}{:}
                \PY{n}{c} \PY{o}{=} \PY{n}{a} \PY{o}{+} \PY{n}{b}
                \PY{n}{a} \PY{o}{=} \PY{n}{b}
                \PY{n}{b} \PY{o}{=} \PY{n}{c}
                \PY{n}{fiblist}\PY{o}{.}\PY{n}{append}\PY{p}{(}\PY{n}{b}\PY{p}{)}
            \PY{k}{return} \PY{n}{fiblist}
\end{Verbatim}


    \begin{Verbatim}[commandchars=\\\{\}]
{\color{incolor}In [{\color{incolor}10}]:} \PY{k}{assert}\PY{p}{(}\PY{n}{fibbo}\PY{p}{(}\PY{l+m+mi}{5}\PY{p}{)}\PY{o}{==}\PY{p}{[}\PY{l+m+mi}{1}\PY{p}{,} \PY{l+m+mi}{1}\PY{p}{,} \PY{l+m+mi}{2}\PY{p}{,} \PY{l+m+mi}{3}\PY{p}{,} \PY{l+m+mi}{5}\PY{p}{]}\PY{p}{)}
\end{Verbatim}


    \hypertarget{exercise-5-strings}{%
\subsubsection{Exercise 5 (Strings)}\label{exercise-5-strings}}

Go through the documentation of string to work through the following
section https://docs.python.org/3/library/string. Combine the following
strings and list (as comma seperated list) into a single string.

    \begin{Verbatim}[commandchars=\\\{\}]
{\color{incolor}In [{\color{incolor}11}]:} \PY{n}{str1} \PY{o}{=} \PY{l+s+s2}{\PYZdq{}}\PY{l+s+s2}{Welcome to Fuse.ai. }\PY{l+s+s2}{\PYZdq{}}
         \PY{n}{str2} \PY{o}{=} \PY{l+s+s2}{\PYZdq{}}\PY{l+s+s2}{This is the First module. }\PY{l+s+s2}{\PYZdq{}}
         \PY{n}{str3} \PY{o}{=} \PY{l+s+s2}{\PYZdq{}}\PY{l+s+s2}{We will cover }\PY{l+s+s2}{\PYZdq{}}
         \PY{n}{modules\PYZus{}list} \PY{o}{=} \PY{p}{[}\PY{l+s+s2}{\PYZdq{}}\PY{l+s+s2}{Python}\PY{l+s+s2}{\PYZdq{}}\PY{p}{,} \PY{l+s+s2}{\PYZdq{}}\PY{l+s+s2}{Python Classes}\PY{l+s+s2}{\PYZdq{}}\PY{p}{,} \PY{l+s+s2}{\PYZdq{}}\PY{l+s+s2}{Numpy}\PY{l+s+s2}{\PYZdq{}}\PY{p}{,} \PY{l+s+s2}{\PYZdq{}}\PY{l+s+s2}{Pandas}\PY{l+s+s2}{\PYZdq{}}\PY{p}{,} \PY{l+s+s2}{\PYZdq{}}\PY{l+s+s2}{Matplotlib}\PY{l+s+s2}{\PYZdq{}}\PY{p}{,} \PY{l+s+s2}{\PYZdq{}}\PY{l+s+s2}{etc.}\PY{l+s+s2}{\PYZdq{}}\PY{p}{]}
         \PY{n}{combinedstr} \PY{o}{=} \PY{k+kc}{None}
         \PY{c+c1}{\PYZsh{} YOUR CODE HERE}
         \PY{n}{combinedstr} \PY{o}{=} \PY{n}{str1} \PY{o}{+} \PY{n}{str2} \PY{o}{+} \PY{n}{str3} \PY{o}{+} \PY{p}{(}\PY{l+s+s2}{\PYZdq{}}\PY{l+s+s2}{, }\PY{l+s+s2}{\PYZdq{}}\PY{p}{)}\PY{o}{.}\PY{n}{join}\PY{p}{(}\PY{n}{modules\PYZus{}list}\PY{p}{)}
\end{Verbatim}


    \begin{Verbatim}[commandchars=\\\{\}]
{\color{incolor}In [{\color{incolor}12}]:} \PY{k}{assert}\PY{p}{(}\PY{n}{combinedstr}\PY{o}{==}\PY{l+s+s1}{\PYZsq{}}\PY{l+s+s1}{Welcome to Fuse.ai. This is the First module. We will cover Python, Python Classes, Numpy, Pandas, Matplotlib, etc.}\PY{l+s+s1}{\PYZsq{}}\PY{p}{)}
\end{Verbatim}


    \hypertarget{exercise-6-set}{%
\subsubsection{Exercise 6 (Set)}\label{exercise-6-set}}

find the common elements of a and b and not in c

    \begin{Verbatim}[commandchars=\\\{\}]
{\color{incolor}In [{\color{incolor}13}]:} \PY{n}{a} \PY{o}{=} \PY{n+nb}{range}\PY{p}{(}\PY{l+m+mi}{10}\PY{p}{,}\PY{l+m+mi}{27}\PY{p}{)}
         \PY{n}{b} \PY{o}{=} \PY{p}{[}\PY{l+m+mi}{1}\PY{p}{,}\PY{l+m+mi}{4}\PY{p}{,}\PY{l+m+mi}{5}\PY{p}{,}\PY{l+m+mi}{8}\PY{p}{,}\PY{l+m+mi}{2}\PY{p}{,}\PY{l+m+mi}{1}\PY{p}{,}\PY{l+m+mi}{9}\PY{p}{,}\PY{l+m+mi}{12}\PY{p}{,}\PY{l+m+mi}{79}\PY{p}{,}\PY{l+m+mi}{35}\PY{p}{,}\PY{l+m+mi}{13}\PY{p}{,}\PY{l+m+mi}{24}\PY{p}{]}
         \PY{n}{c} \PY{o}{=} \PY{p}{(}\PY{l+m+mi}{3}\PY{p}{,}\PY{l+m+mi}{7}\PY{p}{,}\PY{l+m+mi}{12}\PY{p}{,}\PY{l+m+mi}{90}\PY{p}{,}\PY{l+m+mi}{45}\PY{p}{)}
         
         \PY{k}{def} \PY{n+nf}{common}\PY{p}{(}\PY{n}{a}\PY{p}{,}\PY{n}{b}\PY{p}{,}\PY{n}{c}\PY{p}{)}\PY{p}{:}
         \PY{c+c1}{\PYZsh{} YOUR CODE HERE}
             \PY{k}{return} \PY{n+nb}{set}\PY{p}{(}\PY{n}{a}\PY{p}{)} \PY{o}{\PYZam{}} \PY{n+nb}{set}\PY{p}{(}\PY{n}{b}\PY{p}{)} \PY{o}{\PYZhy{}} \PY{n+nb}{set}\PY{p}{(}\PY{n}{c}\PY{p}{)}
\end{Verbatim}


    \begin{Verbatim}[commandchars=\\\{\}]
{\color{incolor}In [{\color{incolor}14}]:} \PY{k}{assert}\PY{p}{(}\PY{n}{common}\PY{p}{(}\PY{n}{a}\PY{p}{,}\PY{n}{b}\PY{p}{,}\PY{n}{c}\PY{p}{)}\PY{o}{==}\PY{p}{\PYZob{}}\PY{l+m+mi}{13}\PY{p}{,}\PY{l+m+mi}{24}\PY{p}{\PYZcb{}}\PY{p}{)}
\end{Verbatim}


    \hypertarget{exercise-7-dictionaries}{%
\subsubsection{Exercise 7
(Dictionaries)}\label{exercise-7-dictionaries}}

Ram went to buy some vegetables in Vegetable market. Write a function to
find his total cost given rams shopping list and price dictionary (price
are in Rupees)

    \begin{Verbatim}[commandchars=\\\{\}]
{\color{incolor}In [{\color{incolor}15}]:} \PY{n}{price} \PY{o}{=} \PY{p}{\PYZob{}}\PY{l+s+s2}{\PYZdq{}}\PY{l+s+s2}{Tomato Big}\PY{l+s+s2}{\PYZdq{}}\PY{p}{:}\PY{l+m+mi}{50}\PY{p}{,}\PY{l+s+s2}{\PYZdq{}}\PY{l+s+s2}{Tomato Small}\PY{l+s+s2}{\PYZdq{}}\PY{p}{:}\PY{l+m+mi}{90}\PY{p}{,}\PY{l+s+s2}{\PYZdq{}}\PY{l+s+s2}{Potato Red}\PY{l+s+s2}{\PYZdq{}}\PY{p}{:}\PY{l+m+mi}{70}\PY{p}{,}\PY{l+s+s2}{\PYZdq{}}\PY{l+s+s2}{Onion Dry}\PY{l+s+s2}{\PYZdq{}}\PY{p}{:}\PY{l+m+mi}{45}\PY{p}{,}\PY{l+s+s2}{\PYZdq{}}\PY{l+s+s2}{Carrot}\PY{l+s+s2}{\PYZdq{}}\PY{p}{:}\PY{l+m+mi}{120}\PY{p}{,}\PY{l+s+s2}{\PYZdq{}}\PY{l+s+s2}{Cabbage}\PY{l+s+s2}{\PYZdq{}}\PY{p}{:}\PY{l+m+mi}{40}\PY{p}{,}\PY{l+s+s2}{\PYZdq{}}\PY{l+s+s2}{Cauli Local}\PY{l+s+s2}{\PYZdq{}}\PY{p}{:}\PY{l+m+mi}{60}\PY{p}{,}\PY{l+s+s2}{\PYZdq{}}\PY{l+s+s2}{Raddish White}\PY{l+s+s2}{\PYZdq{}}\PY{p}{:}\PY{l+m+mi}{40}\PY{p}{,}\PY{l+s+s2}{\PYZdq{}}\PY{l+s+s2}{Brinjal Long}\PY{l+s+s2}{\PYZdq{}}\PY{p}{:}\PY{l+m+mi}{40}\PY{p}{,}\PY{l+s+s2}{\PYZdq{}}\PY{l+s+s2}{Brinjal Round}\PY{l+s+s2}{\PYZdq{}}\PY{p}{:}\PY{l+m+mi}{50}\PY{p}{,}\PY{l+s+s2}{\PYZdq{}}\PY{l+s+s2}{Cow pea}\PY{l+s+s2}{\PYZdq{}}\PY{p}{:}\PY{l+m+mi}{90}\PY{p}{,}\PY{l+s+s2}{\PYZdq{}}\PY{l+s+s2}{French Bean}\PY{l+s+s2}{\PYZdq{}}\PY{p}{:}\PY{l+m+mi}{60}\PY{p}{,}\PY{l+s+s2}{\PYZdq{}}\PY{l+s+s2}{Soyabean Green}\PY{l+s+s2}{\PYZdq{}}\PY{p}{:}\PY{l+m+mi}{80}\PY{p}{,}\PY{l+s+s2}{\PYZdq{}}\PY{l+s+s2}{Bitter Gourd}\PY{l+s+s2}{\PYZdq{}}\PY{p}{:}\PY{l+m+mi}{90}\PY{p}{,}\PY{l+s+s2}{\PYZdq{}}\PY{l+s+s2}{Bottle Gourd}\PY{l+s+s2}{\PYZdq{}}\PY{p}{:}\PY{l+m+mi}{40}\PY{p}{,}\PY{l+s+s2}{\PYZdq{}}\PY{l+s+s2}{Pointed Gourd}\PY{l+s+s2}{\PYZdq{}}\PY{p}{:}\PY{l+m+mi}{50}\PY{p}{,}\PY{l+s+s2}{\PYZdq{}}\PY{l+s+s2}{Smooth Gourd}\PY{l+s+s2}{\PYZdq{}}\PY{p}{:}\PY{l+m+mi}{60}\PY{p}{,}\PY{l+s+s2}{\PYZdq{}}\PY{l+s+s2}{Pumpkin}\PY{l+s+s2}{\PYZdq{}}\PY{p}{:}\PY{l+m+mi}{40}\PY{p}{,}\PY{l+s+s2}{\PYZdq{}}\PY{l+s+s2}{Squash}\PY{l+s+s2}{\PYZdq{}}\PY{p}{:}\PY{l+m+mi}{60}\PY{p}{,}\PY{l+s+s2}{\PYZdq{}}\PY{l+s+s2}{Okara}\PY{l+s+s2}{\PYZdq{}}\PY{p}{:}\PY{l+m+mi}{90}\PY{p}{,}\PY{l+s+s2}{\PYZdq{}}\PY{l+s+s2}{Barela}\PY{l+s+s2}{\PYZdq{}}\PY{p}{:}\PY{l+m+mi}{60}\PY{p}{,}\PY{l+s+s2}{\PYZdq{}}\PY{l+s+s2}{Arum}\PY{l+s+s2}{\PYZdq{}}\PY{p}{:}\PY{l+m+mi}{60}\PY{p}{,}\PY{l+s+s2}{\PYZdq{}}\PY{l+s+s2}{Christophine}\PY{l+s+s2}{\PYZdq{}}\PY{p}{:}\PY{l+m+mi}{20}\PY{p}{,}\PY{l+s+s2}{\PYZdq{}}\PY{l+s+s2}{Brd Leaf Mustard}\PY{l+s+s2}{\PYZdq{}}\PY{p}{:}\PY{l+m+mi}{60}\PY{p}{,}\PY{l+s+s2}{\PYZdq{}}\PY{l+s+s2}{Spinach Leaf}\PY{l+s+s2}{\PYZdq{}}\PY{p}{:}\PY{l+m+mi}{100}\PY{p}{,}\PY{l+s+s2}{\PYZdq{}}\PY{l+s+s2}{Cress Leaf}\PY{l+s+s2}{\PYZdq{}}\PY{p}{:}\PY{l+m+mi}{100}\PY{p}{,}\PY{l+s+s2}{\PYZdq{}}\PY{l+s+s2}{Mustard Leaf}\PY{l+s+s2}{\PYZdq{}}\PY{p}{:}\PY{l+m+mi}{60}\PY{p}{,}\PY{l+s+s2}{\PYZdq{}}\PY{l+s+s2}{Fenugreek Leaf}\PY{l+s+s2}{\PYZdq{}}\PY{p}{:}\PY{l+m+mi}{100}\PY{p}{,}\PY{l+s+s2}{\PYZdq{}}\PY{l+s+s2}{Onion Green}\PY{l+s+s2}{\PYZdq{}}\PY{p}{:}\PY{l+m+mi}{110}\PY{p}{,}\PY{l+s+s2}{\PYZdq{}}\PY{l+s+s2}{Mushroom}\PY{l+s+s2}{\PYZdq{}}\PY{p}{:}\PY{l+m+mi}{180}\PY{p}{,}\PY{l+s+s2}{\PYZdq{}}\PY{l+s+s2}{Neuro}\PY{l+s+s2}{\PYZdq{}}\PY{p}{:}\PY{l+m+mi}{80}\PY{p}{,}\PY{l+s+s2}{\PYZdq{}}\PY{l+s+s2}{Sugarbeet}\PY{l+s+s2}{\PYZdq{}}\PY{p}{:}\PY{l+m+mi}{90}\PY{p}{,}\PY{l+s+s2}{\PYZdq{}}\PY{l+s+s2}{Lettuce}\PY{l+s+s2}{\PYZdq{}}\PY{p}{:}\PY{l+m+mi}{60}\PY{p}{,}\PY{l+s+s2}{\PYZdq{}}\PY{l+s+s2}{Celery}\PY{l+s+s2}{\PYZdq{}}\PY{p}{:}\PY{l+m+mi}{180}\PY{p}{,}\PY{l+s+s2}{\PYZdq{}}\PY{l+s+s2}{Parseley}\PY{l+s+s2}{\PYZdq{}}\PY{p}{:}\PY{l+m+mi}{180}\PY{p}{,}\PY{l+s+s2}{\PYZdq{}}\PY{l+s+s2}{Fennel Leaf}\PY{l+s+s2}{\PYZdq{}}\PY{p}{:}\PY{l+m+mi}{100}\PY{p}{,}\PY{l+s+s2}{\PYZdq{}}\PY{l+s+s2}{Mint}\PY{l+s+s2}{\PYZdq{}}\PY{p}{:}\PY{l+m+mi}{180}\PY{p}{,}\PY{l+s+s2}{\PYZdq{}}\PY{l+s+s2}{Turnip A}\PY{l+s+s2}{\PYZdq{}}\PY{p}{:}\PY{l+m+mi}{60}\PY{p}{,}\PY{l+s+s2}{\PYZdq{}}\PY{l+s+s2}{Tamarind}\PY{l+s+s2}{\PYZdq{}}\PY{p}{:}\PY{l+m+mi}{150}\PY{p}{,}\PY{l+s+s2}{\PYZdq{}}\PY{l+s+s2}{Bamboo Shoot}\PY{l+s+s2}{\PYZdq{}}\PY{p}{:}\PY{l+m+mi}{80}\PY{p}{,}\PY{l+s+s2}{\PYZdq{}}\PY{l+s+s2}{Tofu}\PY{l+s+s2}{\PYZdq{}}\PY{p}{:}\PY{l+m+mi}{100}\PY{p}{,}\PY{l+s+s2}{\PYZdq{}}\PY{l+s+s2}{Gundruk}\PY{l+s+s2}{\PYZdq{}}\PY{p}{:}\PY{l+m+mi}{220}\PY{p}{,}\PY{l+s+s2}{\PYZdq{}}\PY{l+s+s2}{Apple}\PY{l+s+s2}{\PYZdq{}}\PY{p}{:}\PY{l+m+mi}{120}\PY{p}{,}\PY{l+s+s2}{\PYZdq{}}\PY{l+s+s2}{Banana}\PY{l+s+s2}{\PYZdq{}}\PY{p}{:}\PY{l+m+mi}{80}\PY{p}{,}\PY{l+s+s2}{\PYZdq{}}\PY{l+s+s2}{Lime}\PY{l+s+s2}{\PYZdq{}}\PY{p}{:}\PY{l+m+mi}{250}\PY{p}{,}\PY{l+s+s2}{\PYZdq{}}\PY{l+s+s2}{Pomegranate}\PY{l+s+s2}{\PYZdq{}}\PY{p}{:}\PY{l+m+mi}{200}\PY{p}{,}\PY{l+s+s2}{\PYZdq{}}\PY{l+s+s2}{Orange}\PY{l+s+s2}{\PYZdq{}}\PY{p}{:}\PY{l+m+mi}{130}\PY{p}{,}\PY{l+s+s2}{\PYZdq{}}\PY{l+s+s2}{Water Melon}\PY{l+s+s2}{\PYZdq{}}\PY{p}{:}\PY{l+m+mi}{60}\PY{p}{,}\PY{l+s+s2}{\PYZdq{}}\PY{l+s+s2}{Sweet Orange}\PY{l+s+s2}{\PYZdq{}}\PY{p}{:}\PY{l+m+mi}{120}\PY{p}{,}\PY{l+s+s2}{\PYZdq{}}\PY{l+s+s2}{Pineapple}\PY{l+s+s2}{\PYZdq{}}\PY{p}{:}\PY{l+m+mi}{120}\PY{p}{,}\PY{l+s+s2}{\PYZdq{}}\PY{l+s+s2}{Cucumber}\PY{l+s+s2}{\PYZdq{}}\PY{p}{:}\PY{l+m+mi}{70}\PY{p}{,}\PY{l+s+s2}{\PYZdq{}}\PY{l+s+s2}{Papaya}\PY{l+s+s2}{\PYZdq{}}\PY{p}{:}\PY{l+m+mi}{80}\PY{p}{,}\PY{l+s+s2}{\PYZdq{}}\PY{l+s+s2}{Guava}\PY{l+s+s2}{\PYZdq{}}\PY{p}{:}\PY{l+m+mi}{60}\PY{p}{,}\PY{l+s+s2}{\PYZdq{}}\PY{l+s+s2}{Mombin}\PY{l+s+s2}{\PYZdq{}}\PY{p}{:}\PY{l+m+mi}{50}\PY{p}{,}\PY{l+s+s2}{\PYZdq{}}\PY{l+s+s2}{Ginger}\PY{l+s+s2}{\PYZdq{}}\PY{p}{:}\PY{l+m+mi}{130}\PY{p}{,}\PY{l+s+s2}{\PYZdq{}}\PY{l+s+s2}{Chilli Dry}\PY{l+s+s2}{\PYZdq{}}\PY{p}{:}\PY{l+m+mi}{250}\PY{p}{,}\PY{l+s+s2}{\PYZdq{}}\PY{l+s+s2}{Chilli Green}\PY{l+s+s2}{\PYZdq{}}\PY{p}{:}\PY{l+m+mi}{90}\PY{p}{,}\PY{l+s+s2}{\PYZdq{}}\PY{l+s+s2}{Capsicum}\PY{l+s+s2}{\PYZdq{}}\PY{p}{:}\PY{l+m+mi}{90}\PY{p}{,}\PY{l+s+s2}{\PYZdq{}}\PY{l+s+s2}{Garlic Green}\PY{l+s+s2}{\PYZdq{}}\PY{p}{:}\PY{l+m+mi}{290}\PY{p}{,}\PY{l+s+s2}{\PYZdq{}}\PY{l+s+s2}{Coriander Green}\PY{l+s+s2}{\PYZdq{}}\PY{p}{:}\PY{l+m+mi}{100}\PY{p}{,}\PY{l+s+s2}{\PYZdq{}}\PY{l+s+s2}{Garlic Dry Chinese}\PY{l+s+s2}{\PYZdq{}}\PY{p}{:}\PY{l+m+mi}{150}\PY{p}{,}\PY{l+s+s2}{\PYZdq{}}\PY{l+s+s2}{Garlic Dry Nepali}\PY{l+s+s2}{\PYZdq{}}\PY{p}{:}\PY{l+m+mi}{130}\PY{p}{,}\PY{l+s+s2}{\PYZdq{}}\PY{l+s+s2}{Fish Fresh}\PY{l+s+s2}{\PYZdq{}}\PY{p}{:}\PY{l+m+mi}{270}\PY{p}{\PYZcb{}}
         
         \PY{k}{def} \PY{n+nf}{totalcost}\PY{p}{(}\PY{n}{rams\PYZus{}list}\PY{p}{)}\PY{p}{:}
         \PY{c+c1}{\PYZsh{} YOUR CODE HERE}
             \PY{n}{total} \PY{o}{=} \PY{l+m+mi}{0}
             \PY{k}{for} \PY{n}{i} \PY{o+ow}{in} \PY{n}{rams\PYZus{}list}\PY{p}{:}
                 \PY{n}{total}\PY{o}{+}\PY{o}{=}\PY{n}{price}\PY{p}{[}\PY{n}{i}\PY{p}{]}
             \PY{k}{return} \PY{n}{total}
\end{Verbatim}


    \begin{Verbatim}[commandchars=\\\{\}]
{\color{incolor}In [{\color{incolor}16}]:} \PY{k}{assert}\PY{p}{(}\PY{n}{totalcost}\PY{p}{(}\PY{p}{[}\PY{l+s+s2}{\PYZdq{}}\PY{l+s+s2}{Tomato Big}\PY{l+s+s2}{\PYZdq{}}\PY{p}{,}\PY{l+s+s2}{\PYZdq{}}\PY{l+s+s2}{Tomato Small}\PY{l+s+s2}{\PYZdq{}}\PY{p}{]}\PY{p}{)}\PY{o}{==}\PY{l+m+mi}{140}\PY{p}{)}
\end{Verbatim}


    \hypertarget{exercise-8-date-time}{%
\subsubsection{Exercise 8 (Date Time)}\label{exercise-8-date-time}}

Python supports datetime module which supplies functons and classes for
manipulating dates and times in both simple and complex ways.
https://docs.python.org/3/library/datetime.html

Import datetime

    \begin{Verbatim}[commandchars=\\\{\}]
{\color{incolor}In [{\color{incolor}17}]:} \PY{c+c1}{\PYZsh{} YOUR CODE HERE}
         \PY{k+kn}{import} \PY{n+nn}{datetime}
\end{Verbatim}


    \hypertarget{exercise-9}{%
\subsubsection{Exercise 9}\label{exercise-9}}

Since we all love holidays. Write a function to test if the day of the
week is saturday or sunday

    \begin{Verbatim}[commandchars=\\\{\}]
{\color{incolor}In [{\color{incolor}18}]:} \PY{k}{def} \PY{n+nf}{isholiday}\PY{p}{(}\PY{n}{date}\PY{p}{)}\PY{p}{:}
         \PY{c+c1}{\PYZsh{} YOUR CODE HERE}
             \PY{k}{return} \PY{n}{date}\PY{o}{.}\PY{n}{weekday}\PY{p}{(}\PY{p}{)}\PY{o}{==}\PY{l+m+mi}{6} \PY{o+ow}{or} \PY{n}{date}\PY{o}{.}\PY{n}{weekday}\PY{p}{(}\PY{p}{)}\PY{o}{==}\PY{l+m+mi}{5}
\end{Verbatim}


    \begin{Verbatim}[commandchars=\\\{\}]
{\color{incolor}In [{\color{incolor}19}]:} \PY{k}{assert}\PY{p}{(}\PY{n}{isholiday}\PY{p}{(}\PY{n}{datetime}\PY{o}{.}\PY{n}{datetime}\PY{p}{(}\PY{l+m+mi}{2018}\PY{p}{,} \PY{l+m+mi}{10}\PY{p}{,} \PY{l+m+mi}{21}\PY{p}{)}\PY{p}{)}\PY{o}{==}\PY{k+kc}{True}\PY{p}{)}
\end{Verbatim}


    \hypertarget{list-comprehension}{%
\subsubsection{List Comprehension}\label{list-comprehension}}

https://docs.python.org/3/tutorial/datastructures.html\#list-comprehensions

Also go through - https://docs.python.org/3/library/functions.html\#zip
- https://docs.python.org/3/library/itertools.html,

as these functions are very useful in different situation

    \begin{Verbatim}[commandchars=\\\{\}]
{\color{incolor}In [{\color{incolor}20}]:} \PY{c+c1}{\PYZsh{} Experiement with List comprehension here}
\end{Verbatim}


    \hypertarget{exercise-10}{%
\subsubsection{Exercise 10}\label{exercise-10}}

Carryout matrix multiplication using list comprehension. Zip function
might come handy for this task

    \begin{Verbatim}[commandchars=\\\{\}]
{\color{incolor}In [{\color{incolor}21}]:} \PY{k}{def} \PY{n+nf}{matmultiply}\PY{p}{(}\PY{n}{A}\PY{p}{,} \PY{n}{B}\PY{p}{)}\PY{p}{:}
             \PY{c+c1}{\PYZsh{} Write Solution in 1 line}
         \PY{c+c1}{\PYZsh{} YOUR CODE HERE}
             \PY{k}{return} \PY{p}{[}\PY{p}{[}\PY{n+nb}{sum}\PY{p}{(}\PY{n}{x}\PY{o}{*}\PY{n}{y} \PY{k}{for} \PY{n}{x}\PY{p}{,}\PY{n}{y} \PY{o+ow}{in} \PY{n+nb}{zip}\PY{p}{(}\PY{n}{row\PYZus{}A}\PY{p}{,}\PY{n}{col\PYZus{}B}\PY{p}{)}\PY{p}{)} \PY{k}{for} \PY{n}{col\PYZus{}B} \PY{o+ow}{in} \PY{n+nb}{zip}\PY{p}{(}\PY{o}{*}\PY{n}{B}\PY{p}{)}\PY{p}{]} \PY{k}{for} \PY{n}{row\PYZus{}A} \PY{o+ow}{in} \PY{n}{A}\PY{p}{]}
\end{Verbatim}



    % Add a bibliography block to the postdoc
    
    
    
    \end{document}
